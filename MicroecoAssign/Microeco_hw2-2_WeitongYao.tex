%% LyX 2.3.2-2 created this file.  For more info, see http://www.lyx.org/.
%% Do not edit unless you really know what you are doing.
\documentclass[12pt,english]{article}
\usepackage[T1]{fontenc}
\usepackage[utf8]{inputenc}
\usepackage{geometry}
\geometry{verbose,bmargin=1.25in,lmargin=1.2in,rmargin=1.2in}
\setlength{\parindent}{0in}
\usepackage{babel}
\usepackage{longtable}
\usepackage{amsmath}
\usepackage{amsthm}
\usepackage{graphicx}
\usepackage{setspace}
\usepackage[unicode=true]
 {hyperref}

\makeatletter

%%%%%%%%%%%%%%%%%%%%%%%%%%%%%% LyX specific LaTeX commands.
%% Because html converters don't know tabularnewline
\providecommand{\tabularnewline}{\\}

%%%%%%%%%%%%%%%%%%%%%%%%%%%%%% Textclass specific LaTeX commands.
\numberwithin{equation}{section}
\numberwithin{figure}{section}

\@ifundefined{date}{}{\date{}}
%%%%%%%%%%%%%%%%%%%%%%%%%%%%%% User specified LaTeX commands.
\usepackage{CJK}
\usepackage{listings}
\usepackage{xcolor}
\lstset{
  %行号
  numbers=left,
  %背景框
  framexleftmargin=10mm,
  frame=none,
  %背景色
  %backgroundcolor=\color[rgb]{1,1,0.76},
  backgroundcolor=\color[RGB]{245,245,244},
  %样式
  keywordstyle=\bf\color{blue},
  identifierstyle=\bf,
  numberstyle=\color[RGB]{0,192,192},
  commentstyle=\it\color[RGB]{0,96,96},
  stringstyle=\rmfamily\slshape\color[RGB]{128,0,0},
  %显示空格
  showstringspaces=false,
  xleftmargin=2em, %边距
  xrightmargin=2em, 
  aboveskip=1em
}

\makeatother

\begin{document}
\begin{CJK}{UTF8}{gbsn}
\title{Does Housing Price Impede the Rising Birth Rate?}
\author{Yao Weitong}

\maketitle
According to the Malthusian theory of population, it’s the idea that
population growth is potentially exponential while the growth of the
food supply is linear. Malthusianism has concluded two “checks” to
impede the population growth: “preventive checks” (moral restraints)
and “positive checks” (disease, starvation, war.etc.). Obviously,
technology development hasn’t been taken into account by Malthusian;
in another words, these “checks” affect little in modern society,
and the population of the world will rise dramatically (and it does!)$^{[1]}$.
However, some people study the relationship between high housing price
and declining birth rate in Japan and they consider the rising housing
price, as cost of raising babies, is a kind of “check” of Malthusiam.
Is it true and general? Based on the dramatically increasing housing
price in China since 2008, I would like to check whether housing prices
do reduce fertility in the long run.

\section{Regression Analysis}

Fertility rates are linked to many factors, such as the cost of raising
children -- rising house prices, education -- especially the ratio
of female educated, women’s labor participation rate, the idea of
``rearing kids for old age'' -- reflected in pension expenditures
and the entire society’s elderly dependency rate. This passage collects
data from the National Bureau of Statistics in mainland China from
2000 to 2017, and supplements it with data from the China Statistical
Yearbook to make a simple multiple linear regression and to make a
simple diagnosis of the model.

\subsection{Variable Descriptions and Multiple Linear Regression Analysis}

\begin{longtable}[c]{ll}
\textbf{Variable} & \textbf{Description}\tabularnewline
{\small{}birthrate} & {\small{}yearly fertility rate}\tabularnewline
{\small{}femalegrow} & {\small{}growth rate of female labor participation rate}\tabularnewline
{\small{}houseprice\_rate} & {\small{}growth rate of residential house prices (rmb per square)}\tabularnewline
{\small{}pensionout} & {\small{}proportion of social pension spending}\tabularnewline
{\small{}devocerate} & {\small{}crude devorce rate}\tabularnewline
{\small{}bachelor} & {\small{}number of undergraduates}\tabularnewline
{\small{}consumpgrow} & {\small{}growth rate of consumer price index}\tabularnewline
{\small{}gdppergrow} & {\small{}growth rate of gdp per capita}\tabularnewline
 & \tabularnewline
\end{longtable}

\[
Birthrate_{t}=\beta_{0}+\text{\ensuremath{\beta}}_{1}housepricerate_{t-1}+\text{\ensuremath{\beta}}_{2}femalegrow_{t-1}+\text{\ensuremath{\beta}}_{3}pensionout_{t-1}
\]
\[
+\text{\ensuremath{\beta}}_{4}devocerate_{t-1}+\text{\ensuremath{\beta}}_{5}bachelor_{t-1}+\text{\ensuremath{\beta}}_{6}consumpgrow_{t-1}+\text{\ensuremath{\beta}}_{7}gdppergrow_{t-1}+e_{t}
\]

\begin{tabular}{c}
\tabularnewline
\end{tabular}

The regression result and assessment of multiple linear model are
as follows:

\begin{tabular}{c}
\tabularnewline
\end{tabular}
\begin{center}
\includegraphics[scale=0.75]{)%`BYT\string~WOLKG9RH}W7D$01D}
\par\end{center}

\begin{center}
\includegraphics[scale=0.75]{$Z3_8A\string~M3$]G5)N22C([7QE}
\par\end{center}

Next, check the multicollinearity of this model:
\begin{center}
\includegraphics[bb=0bp 0bp 583bp 496bp,scale=0.65]{已粘贴1}
\par\end{center}

\begin{center}
\includegraphics[scale=0.65]{已粘贴2}
\par\end{center}

Obviously, there is a problem of multicollinearity in this linear
regression. To improve this modle, the passage uses ridge regression
instead of simple multiple linear regression.

\subsection{Ridge Regression$^{[2,3]}$}

The logit that ridge regression can solve multicollinearity problem
is to reduce the variance of parameter estimators, at the cost of
sacrifying unbiasness. The general principle of ridge regression is
to add a constraint to the parameter after the cost function, based
on the MLE. This constraint is called a \textbf{regularizer}. The
cost function of ridge regression is

\[
J(\theta)=\frac{1}{n}\sum_{i=1}^{n}(Y_{i}-BX_{i})^{2}+\lambda||B||^{2}=MSE(\theta)+\lambda\sum_{i=1}^{m}\theta_{i}^{2}
\]

where $B$ is a coefficient vector excluding intercept term; $\theta$
is a n+1 vector including intercept term $\theta_{0}$; $n$ is the
sample size and $m$ is the number of independent variables. By making
the gradient of $J(\theta)$ equal to 0, the optimal solution to minimize
$J(\theta)$ is

\[
\theta=(X^{T}X+\lambda I)^{-1}(X^{T}Y)
\]

Here are the ridge regression results; the significance of regression
coefficient is significantly higher than that of multiple linear regression.
\begin{center}
\includegraphics[scale=0.6]{已粘贴3}
\par\end{center}

\begin{center}
\includegraphics[scale=0.75]{)3B0SM_Q5U4F}3Y]%4%O0{J}
\par\end{center}

\subsection{Explanation of Results}

The results show that in the long run, the feritility rate is only
negatively correlated with the divorce rate and education. The influence
of devorce rate is greater in an economical sense, and the statistical
results are more significant, too. It has nothing to do with the rate
of house price appreciation. Further, assuming that there is a strong
correlation between the increase in house prices and the divorce rate,
I further run two regression and find that the results are either
insignificant or significant but weakly correlated. Although the cost
of raising children is rising when house prices rise, the regression
shows that women's labor participation rate and GDP growth explain
the decline in child dependency ratio more. Therefore, the decline
in fertility in the long-term cannot be attributed to the rise in
housing prices; or said, the house as a fixed asset is the main collateral
for developers and corporate borrowing. In order to maintain the normal
economy operation, the government would like maintain stability rather
than suppress it. Directly speaking, it is an inevitable trend that
house prices are at high prices, but people wouldn't choose to be
DINK because of this. It is obvious that the house only temporarily
sway the decision of fertility, but it cannot affect fertility in
the long run.

\section{Conclusion and Other Evidences}

Other evidences can show why they are not exactly negatively correlated.
First is about the two-sided impacts. According to\textit{ Lisa J.Dettling
and Melissa S.Kearne}y$^{[4]}$, current house prices can lead to
a negative effect on the fertility rates among renters but positive
effect on births among homeowners during the current period and these
two effects can offset each other. Second, children-rearing cost is
a bridge linking housing prices and fertility rate, however, the young
dependency rate is falling in recent year, from the data collected
from National Bureau of Statistics. In other words, if having a children
but not raising them, then the impact of housing prices on fertility
rate will not be established. Third one is the elasticity issue. By
\textit{Creina Day and Ross Guest}$^{[5]}$, the key is the level
of substitution effect of rising female wages on fertility depends
on how important housing is as a cost of children; price elasticity
of housing supply is low in Asia, so the increasing wage leads to
decreasing birth rate. Things are different in western countries,
where housing supply is relatively price elastic.

\section*{References}

\begin{onehalfspace}
{\small{}{[}1{]}}\textit{\small{} Malthusiam}{\small{}, wikipedia.}{\small\par}

{\small{}{[}2{]} }\textit{\small{}M. El-Dereny and N.I. Rashwan.}{\small{}
2011. Solving Multicollinearity Problem Using Ridge Regression Models.
Int J. Contemp. Math. Sciences, Vol.6,2001, no.12,585-600.}{\small\par}

{\small{}{[}3{]} \href{https://www.cnblogs.com/Belter/p/8536939.html}{【机器学习】正则化的线性回归 —— 岭回归与Lasso回归}}{\small\par}

{\small{}{[}4{]} }\textit{\small{}Lisa J.Dettling and Melissa S.Kearney}{\small{}.2016.
Fertility and female wages: A new link via house prices. Economic
Modelling.}{\small\par}

{\small{}{[}5{]} }\textit{\small{}Creina Day and Ross Guest. }{\small{}2014.
House prices and birth rates: The impact of the real estate market
on the decision to have a baby. Journal of Public Economics}{\small\par}
\end{onehalfspace}

\end{CJK}
\end{document}
