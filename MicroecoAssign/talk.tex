\documentclass{beamer}	% Compile at least twice!
%\setbeamertemplate{navigation symbols}{}
\usetheme{Warsaw}
%\useinnertheme{rectangles}
%\useoutertheme{infolines}
\useoutertheme[title,section,subsection=true]{smoothbars}
 
% -------------------
% Packages
% -------------------
\usepackage{
	amsmath,			% Math Environments
	amssymb,			% Extended Symbols
	enumerate,		    % Enumerate Environments
	graphicx,			% Include Images
	lastpage,			% Reference Lastpage
	multicol,			% Use Multi-columns
	multirow,			% Use Multi-rows
	pifont,			    % For Checkmarks
	stmaryrd			% For brackets
}
\usepackage[english]{babel}
\usepackage{hyperref}
\usepackage{subfigure}
\usepackage{geometry}
\usepackage{makecell}

% -------------------
% Colors
% -------------------
\definecolor{UniOrange}{RGB}{212,69,0}
\definecolor{UniGray}{RGB}{62,61,60}
%\definecolor{UniRed}{HTML}{B31B1B}
%\definecolor{UniGray}{HTML}{222222}
\setbeamercolor{title}{fg=UniGray}
\setbeamercolor{frametitle}{fg=UniOrange}
\setbeamercolor{structure}{fg=UniOrange}
\setbeamercolor{section in head/foot}{bg=UniGray}
\setbeamercolor{author in head/foot}{bg=UniGray}
\setbeamercolor{date in head/foot}{fg=UniGray}
\setbeamercolor{structure}{fg=UniOrange}
\setbeamercolor{local structure}{fg=black}
\beamersetuncovermixins{\opaqueness<1>{0}}{\opaqueness<2->{15}}


% -------------------
% Fonts & Layout
% -------------------
\useinnertheme{default}
\usefonttheme{serif}
\usepackage{palatino}
\setbeamerfont{title like}{shape=\scshape}
\setbeamertemplate{itemize items}[circle]
%\setbeamertemplate{enumerate items}[default]


% -------------------
% Commands
% -------------------

% Special Characters
\newcommand{\N}{\mathbb{N}}
\newcommand{\Z}{\mathbb{Z}}
\newcommand{\Q}{\mathbb{Q}}
\newcommand{\R}{\mathbb{R}}
\newcommand{\C}{\mathbb{C}}

% Math Operators
\DeclareMathOperator{\im}{im}
\DeclareMathOperator{\Span}{span}

% Special Commands
\newcommand{\pf}{\noindent\emph{Proof. }}
\newcommand{\ds}{\displaystyle}
\newcommand{\defeq}{\stackrel{\text{def}}{=}}
\newcommand{\ov}[1]{\overline{#1}}
\newcommand{\ma}[1]{\stackrel{#1}{\longrightarrow}}
\newcommand{\twomatrix}[4]{\begin{pmatrix} #1 & #2 \\ #3 & #4 \end{pmatrix}}


% -------------------
% Tikz & PGF
% -------------------
\usepackage{tikz}
\usepackage{tikz-cd}
\usetikzlibrary{
	calc,
	decorations.pathmorphing,
	matrix,arrows,
	positioning,
	shapes.geometric
}
\usepackage{pgfplots}
\pgfplotsset{compat=newest}


% -------------------
% Theorem Environments
% -------------------
\theoremstyle{plain}
\newtheorem{thm}{Theorem}[section]
\newtheorem{prop}{Proposition}[section]
\newtheorem{lem}{Lemma}[section]
\newtheorem{cor}{Corollary}[section]
\theoremstyle{definition}
\newtheorem{ex}{Example}[section]
\newtheorem{nex}{Non-Example}[section]
\newtheorem{dfn}{Definition}[section]
\theoremstyle{remark}
\newtheorem{rem}{Remark}[section] 
\numberwithin{equation}{section}


% =================================================================================================================
%                                                       Title Page
% =================================================================================================================
\title{\textcolor{white}{Approaches to Estimating Aggregate Demand for Reserve Balances}}
\author{Presenter: Yao Weitong}
\date{November 8, 2019} 


% -------------------
% Content
% -------------------
\begin{document}

% Title Page
\begin{frame}
\titlepage
\end{frame}

% Contents
\section{Contents}
\begin{frame}{size = \large}
	\begin{enumerate}[1.]%\setlength{\itemsep}{想要的值如15pt } 
	\item Summary
	\item Approaches for Estimation
		\begin{itemize}\setlength{\itemsep}{3pt} 
		\item Building the Point Estimate
		\item Approach \uppercase\expandafter{\romannumeral1} : Stratified Sampling for Sampling Error
		\item Approach \uppercase\expandafter{\romannumeral2}: Bootstrap for Non-Sampling Error
		\end{itemize} 
	\item Conclusion
	\item Short Comings
	\end{enumerate}  \vspace{0.5cm}
\end{frame}


% -------------------
% Part 1: Summary
% -------------------
\section{Summary}

\begin{frame}
\frametitle{\textcolor{white}{Summary}}
\small 
The August 2019 Senior Financial Officer Surveys(SFOS) asked 77 banks to report the approximate lowest level of reserve balances(LCLoR), given the constellation of short-term interest rates.\\

\vbox{} 
Based on statistical methods applied in \href{https://www.federalreserve.gov/econres/notes/feds-notes/estimating-system-demand-for-reserve-balances-using-the-2018-senior-financial-officer-survey-20190409.htm}{\textcolor{UniOrange}{prior FEDS Notes}}, a point estimate for aggregate demanad, accouting for sampling and non-sampling error, is \$800 billion, ranging between \$712 billion and \$919 billion.\\

\vbox{}
These banks represented a range of asset sizes and business models, and the short-term interest rates include IOER, EFFR, SOFR, OBFR and Treasury bill rate.
\end{frame}\vspace{0.5cm}


% Why Does the Demand for Reserve Balances Matters?
\begin{frame}
\frametitle{\textcolor{white}{Why Does the Demand for Reserve Balances Matter?}}
\scriptsize

\begin{figure}[htbp]
	\subfigure{
	\begin{minipage}[t]{0.5\linewidth}
	\includegraphics[width=2.2in]{1.png}
	\end{minipage}%
	}%
	\subfigure{
	\begin{minipage}[t]{0.5\linewidth}
	\includegraphics[width=2.2in]{2.png}
	\end{minipage}
	}%
\end{figure}

An ample supply of reserves made the interest rate insensitive to OMO. Instead, overnight interest rates moved in line with administered rates.\\

\vbox{}
Reserve demand is driven by many factors and it changes continually, so reserve demand and  monetary policies codetermine what the interest rate is. \href{https://www.newyorkfed.org/newsevents/speeches/2019/log190417}{\textcolor{UniOrange}{(''Observation on Implementing Monetary Policy in an Ample-Reserves Regime'', Lorie K.Logan)}}

\end{frame}\vspace{0.5cm}




% -------------------
% Part 2: Approaches
% -------------------
\section{Approaches}

% 2.1 Building the Point Estimation
\begin{frame}
\frametitle{\textcolor{white}{Building the Point Estimation}}


\small\textbf{Comparision ( just like ''Descriptive Statistics''):}

\renewcommand\arraystretch{1.5}
\begin{table} 
	\scriptsize
	\begin{center}  
	\begin{tabular}{|l|l|l|l|}  
	\hline  
	      & SFOS Respondent Banks & Non-SFOS Banks & total \\ 
 	\hline  
    LCLoR & 652 &  \multicolumn{2}{c|}{to be estimated}\\ 
    \hline  
    Total Reserve Bal. & 1152 & 334& 1486 \\ 
    \hline
    Total Asset & 12943 & 6840 & 19783\\  
    \hline 
    \multirow{4}*{Categories}  & \multicolumn{3}{l|}{(1)U.S. Global Systemically Important Banks(G-SIBs)}\\  
    \cline{2-4}
    ~ & \multicolumn{3}{l|}{(2)Large Domestic Banks}\\  
    \cline{2-4}
    ~ & \multicolumn{3}{l|}{(3)Small Domestic Banks}\\  
    \cline{2-4}
    ~ & \multicolumn{3}{l|}{(4)Foreign Banking Organizations(FBOs)}\\
    \hline  
	\end{tabular}  
	\end{center}  
\end{table}

\scriptsize 77 SFOS banks have $\frac{3}{4}$ of reserve holdings, which is \$652 billion. The number of Non-SFOS banks is around 5200 but their size is only half of SFOS banks.

\end{frame}\vspace{0.5cm}



% Formula & Uncertainty
\begin{frame}
\frametitle{\textcolor{white}{Building the Point Estimation}}
\small
\textbf{Formulas for Estimation:}
\begin{itemize}
	\scriptsize
	\item[(1)] SFOS category ratio = aggregate SFOS LCLoR / aggregate SFOS total assets
	\item[(2)]
	NonSFOS LCLoR estimate$_{i}=$ SFOS category ratio $\times \sum_{i}$ total 	assets\\
	\item[(3)]aggregate point estimate = $\sum$ $LCLoR_i$ for SFOS + estimate of 	$LCLoR$ for NonSFOS
	\item[*]The weights are relative to their size! Think about the Fed Reserve Ratio!
\end{itemize}

\vbox{}

\textbf{Questions needed to be solved:}
\begin{itemize}
	\scriptsize
	\item[(1)]\textbf{Sampling error} : comes from random selection of SFOS banks. 
	\item[(2)]\textbf{Non-Sampling error} : this sample(SFOS banks) is not so representative.
\end{itemize}

\vbox{}
\textbf{$\to$ Stratified Sampling , Multiple Imputation(Bootstrap)}

\end{frame}

% 2.2 Stratified Sampling
\begin{frame}
\frametitle{\textcolor{white}{Approach \uppercase\expandafter{\romannumeral1} : Stratified Sampling for Sampling Error}}
\small

\textbf{Setup: (''Sampling Techniques, Cochran, Chapter5 Section 5.1-5.4'')}
\begin{itemize}
	\scriptsize
	\item[(1)]There are 4 categories, sorting into L = 4 layers.
	\item[(2)]The size of population $N$ is 77. Suppose $N_{j}$ is number of units in $j th$ layer, thus $N_{1} + N_{2} + N_{3} + N_{4} = N = 77.$ 
	\item[(3)]For each layer, random selection is applied for obtaining one series of observations as samples (a subset of all LCLoRs in each category). Suppose $n_{j}$ is number of units in sample for $j th$ layer, $y_{ij}$ is the $ith$ unit's value of LCLoR drawn from $j th$ layer.
	\item[(4)]Stratum weight for $j th$ layer:  
	
	\begin{equation}
	W_{j} = \frac{\text{total assets in j th category in population}}{\text{total population assets}}
	\end{equation}
	
	true mean for $j th$ layer:
	\begin{equation}
	\bar{Y}_{j}=\frac{\sum_{i=1}^{N_{j}} y_{i j}}{N_{j}}
	\end{equation}
	
\end{itemize}

\end{frame}




\begin{frame}
\frametitle{\textcolor{white}{Approach \uppercase\expandafter{\romannumeral1} : Stratified Sampling for Sampling Error}}
\small
\textbf{Setup: (Continued)}

\begin{itemize}
	\scriptsize
	\item[(4)]sample mean for $j th$ layer:
	\begin{equation}
	\bar{y}_{j}=\frac{\sum_{i=1}^{n_{j}} y_{i j}}{n_{j}}
	\end{equation}
	
	true variance for $j th$ layer:
	\begin{equation}
	S_{j}^{2}=\frac{\sum_{i=1}^{N_{j}}\left(y_{i j}-\bar{Y}_{j}\right)^{2}}{N_{j}-1}
	\end{equation}
\end{itemize}

\textbf{Properties}

\begin{itemize}
	\scriptsize
	\item If in every layer the sample estimate $\bar{y}_{j}$ is unbiased, then population mean per unit $\bar{y}_{s t}$ is an unbiased estimate of population mean $\bar{Y}$ \textbf{(mean of aggregate LCLoR for SFOC!)}: 
	\begin{equation}
	\bar{y}_{s t}=\frac{\sum_{j=1}^{L} N_{j} \bar{y}_{j}}{N}=\sum_{j=1}^{L} W_{j} \bar{y}_{j} \approx \bar{Y}
	\end{equation}
	
\end{itemize}

\end{frame}

\begin{frame}

\frametitle{\textcolor{white}{Approach \uppercase\expandafter{\romannumeral1} : Stratified Sampling for Sampling Error}}
\small
\textbf{Properties: (Continued)}

\begin{itemize}
	\scriptsize
	
	\item For stratified sampling, variance of $\bar{y}_{s t}$ and its approximation $s^{2}\left(\bar{y}_{s t}\right)$ \textbf{(now we know the standard error!)}:
	\begin{equation}
	V\left(\bar{y}_{s t}\right)=\frac{1}{N^{2}} \sum_{j=1}^{L} N_{j}\left(N_{j}-n_{j}\right) \frac{S_{j}^{2}}{n_{j}}=\sum_{j=1}^{L} W_{j}^{2} \frac{S_{j}^{2}}{n_{j}}\left(1-\frac{n_{j}}{N_{j}}\right)
	\end{equation}
	
	\begin{equation}
s^{2}\left(\bar{y}_{s t}\right)=\sum_{j=1}^{L} \frac{W_{j}^{2} s_{j}^{2}}{n_{j}}-\sum_{j=1}^{L} \frac{W_{j} s_{j}^{2}}{N}
	\end{equation}
\end{itemize}

\end{frame}

\begin{frame}

\frametitle{\textcolor{white}{Approach \uppercase\expandafter{\romannumeral1} : Stratified Sampling for Sampling Error}}
\small

\textbf{Estimation: \href{https://pdfs.semanticscholar.org/af4b/3d01c44d7a404f64aa6b8da49cdbc8cb0759.pdf}{\textcolor{UniOrange}{(df part comes from F.E.Satterthwaite)}}}

\begin{itemize}
	\scriptsize
	\item[*] Assumption (1) : $\bar{y}_{s t} \sim $Normal Distribution, so MSE$^{2} \sim \chi^{2}$ (and variance of $\chi^{2}$ would be a good approximation for variance of Complex Estimate)
	\item[*] Assumption (2) : Compute standard error $s(\bar{y}_{s t})$ by Student t-statistics because of only a few degrees of freedom
	\item 
	
	\begin{equation}
s^{2}\left(\bar{y}_{s t}\right)=\frac{1}{N^{2}} \sum_{j=1}^{L} g_{j} s_{j}^{2}, \quad \text { where } g_{j}=\frac{N_{j}\left(N_{j}-n_{j}\right)}{n_{j}}
	\end{equation}
	
	the degree of freedom is :
	\begin{equation}
df=\frac{\left(\sum g_{j} s_{j}^{2}\right)^{2}}{\sum \frac{g_{j}^{2} s_{j}^{4}}{n_{j}-1}}
    \end{equation}
    and the result for range of aggregate LCLoR is:
    \begin{equation}
\bar{y}_{s t} \pm t s\left(\bar{y}_{s t}\right)
	\end{equation}
    
\end{itemize}

\end{frame}



% 2.3 Bootstrap
\begin{frame}

\frametitle{\textcolor{white}{Approach \uppercase\expandafter{\romannumeral2} : Bootstrap for Non-Sampling Error}}

\small
\textbf{Introduction of Bootstrap:}
\begin{itemize}\setlength{\itemsep}{1pt} 
	\scriptsize

	\item[*]\textbf{resampling to rebuild a new sample, which can represent the distribution of original sample}
	\item[(1)]\textbf{Characteristics :} 
 	\begin{itemize}\setlength{\itemsep}{1pt} 
 	\scriptsize
	\item Distribution assumption is not necessary
 	\item Returning drawn observations to the data sample after they have been chosen \quad (some observations may repeat themselves, but it can ensure identical sample size!)
 	\end{itemize}

 	\item[(2)]\textbf{Service Condition :} small data set, or difficult to classify
 	
 	\item[(3)]\textbf{Limitation :} different drawing leads different distribution
\end{itemize}

\textbf{Relative Formula:}\\
\begin{itemize}\setlength{\itemsep}{1pt} 
\scriptsize
\item[]
 NonSFOS LCLoR estimate$_{i}=$ \textcolor{blue}{SFOS category ratio *} $\times \sum_{i}$ total 	assets\\
\item[]
Bootstrap can derive a normal distribution for estimating \textcolor{blue}{ SFOS category ratio *}
\end{itemize}

\end{frame}




\begin{frame}
\frametitle{\textcolor{white}{Approach \uppercase\expandafter{\romannumeral2} : Bootstrap for Non-Sampling Error}}
\scriptsize
''by multiplying individual non-SFOS bank's total assets..., but rather is a randomly selected survey response to total asset ratio for an individual SFOS bank in the appropriate category...''\\
\vbox{}

\small \textbf{Procedures:}
\begin{itemize}
\scriptsize
\item[Step 1] 
Suppose the individual SFOS ratio ($\frac{\text{individual SFOS LCLoR}}{\text{individual SFOS asset}}$) is [y1, y2, ..., y77], and they all follow an unknown distribution (iid). 

\item[Step 2]
For the $i$ th sampling, we resample 10 times, noted as $X*_{i} = [X*_{i 1}, X*_{i 2}, ...,X*_{i 10}]$, so the size of each sample is 10. \\
We use 10-units sample to calculate individual SFOS ratio, noted as $\hat{Y}_{i}$

\item[Step 3]
Repeat Step 2 for 1000 times, we can obtain 1000 estimates for individual SFOS ratio, and the number is large enough to build a normal distribution (central limit theorem):
\begin{equation}
\bar{Y} \approx \frac{1}{1000} \sum_{i=1}^{1000} \hat{Y}_{i} \quad var(Y) \approx \frac{1}{1000-1} \sum_{i=1}^{1000} (\hat{Y}_{i}-\bar{Y}_{i})^{2}
\end{equation}

\end{itemize}
\end{frame}


% -------------------
% Part 3 : Conclusion
% -------------------
\section{Conclusion}
\begin{frame}
\frametitle{\textcolor{white}{Conclusions}}
\small
Accounting for sampling error and non-sampling error, the point estimate for aggregate reserve demand is \$800 billion, ranging from \$712 billion to \$919 billion.

\begin{figure}[htbp] 
\centering
\includegraphics[width=4in]{3.png} 
\end{figure} 

\end{frame}

% -------------------
% Part 4: Short Comings 
% -------------------
\section{Short Comings}
\begin{frame}
\frametitle{\textcolor{white}{Short Comings}}

\small
\textbf{Short Comings: may be underestimated!}
\begin{itemize}
\item[1.] Estimating non-sampling error for G-SIB and FBO.
\item[2.] Relying on data reported at a point in time.\\
(the asset data are as of March, 2019, while SFOS was conduted in Aug, 2019)
\item[3.] Assuming perfect money market efficiency and there are no additional factors affecting supply side.
\end{itemize}

\end{frame}


% Questions
\begin{frame}
\begin{center} {\bfseries \Large Thanks for Your Attention} \end{center}
\vbox{}
\begin{center} {\bfseries \Large Q \& A ?} \end{center}
\end{frame}


\end{document}